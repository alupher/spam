\documentclass[preprint]{acm_proc_article-sp}
%\documentclass[preprint]{sig-alternate}
\usepackage{url}
\usepackage{graphicx,subfigure}
\usepackage{xspace}

\newcommand{\ie}{{\em i.e.,}~}
\newcommand{\eg}{{\em e.g.,}~}

\newenvironment{denseitemize}{
\begin{itemize}[topsep=2pt, partopsep=0pt, leftmargin=1.5em]
  \setlength{\itemsep}{4pt}
  \setlength{\parskip}{0pt}
  \setlength{\parsep}{0pt}
}{\end{itemize}}

\newcommand{\eat}[1]{}

\begin{document}

\title{CS 294-1 Project Proposal: \\
Detecting Spam on Social Networking Sites}

\numberofauthors{3}
\author{
Antonio Lupher,
Cliff Engle,
Reynold Xin\\\\
\texttt{\{alupher, cengle, rxin\}@cs.berkeley.edu}
}

\maketitle

\section{Problem}

Most social networks of any significant size see constant spam, scams
and phishing attacks. The nature of these attacks can be quite diverse
and difficult detect. Marketers can spam members with unwanted
advertisements, fraudsters lure users with advance fee frauds and
other confidence tricks, while others attempt to steal user
information by directing users to external phishing pages. To make
matters more difficult, a site with global reach sees communication
among its members in a number foreign languages with varying levels
of ability. This means that much benign content shares characteristics
like misspellings, awkward phrases, etc. that might have made certain
types of common frauds and spam more easy to distinguish on US-based
(or English-language) sites.

\section{Approach}

This project will examine in detail the types of malicious and benign
content that are encountered on social networks by analyzing
experimental data available from InterPals, an international social network
for cultural exchange and language practice. For example, the site
attracts a wide variety of financial scams, ranging from Nigerian
"419" scams to romance scams. Another prevalent problem is spam with
links to third-party websites, directing users to various porn/webcam
sites, phishing sites or various untrustworthy online marketplaces.

We will then examine various methods of detecting and preventing abuse
on the site, including those measures that have already been taken
(e.g. various heuristics including IP/location anomaly detection,
frequency capping, duplicate account detection, etc.). However, the
main focus will be on mining experimental data from the site and using
features derived from this data to build and evaluate classifiers to
detect unwanted behavior programmatically. The large volume of data
available to us will provide a unique perspective both on the types of
malicious content that exist on such sites as well as on the
effectiveness of classifier/learning-based approaches to identifying
these activities.

\section{Data sets}

We plan to make heavy use of our unrestricted access to the data of
InterPals, which has over 1.2 million active members. This data
includes a corpus of 90 million private messages and another 1.5
million messages that have been labeled as spam by users. Other data
includes 40 million or so "wall" comments, 5 million photos, and 8
million photo comments. 

\section{Techniques}

Using this data, we plan to first identify the most prevalent types of 
malicious activities on the site. We will investigate various machine learning 
techniques to automatically detect these activities, including sampling, clustering and 
classifiers. Sampling and clustering, in addition to the user-labeled and moderator-verified 
spam corpus, will help us identify training data. 

Classifiers that we plan to explore 
include: 

\begin{itemize}
\item Gradient boosting
\item Naive Bayes
\item SVMs 
\item Decision trees
\end{itemize}

We will then test the classifiers on new site data to evaluate their performance.

One important part of this project will be to identify features of undesirable activity that are useful in classification. We plan to examine:
\begin{itemize}
\item Profile and message data (keywords, n-grams, age, sex, registration date, amount of data in profile, etc.)
\item User info (geographical location, IPs, browser settings)
\item Complaints and user labeling of spam
\item Profile photographs
\item User reputation, social graph
\end{itemize}

\section{Frameworks}
We hope to implement the detection algorithms using Spark, an
in-memory distributed computing framework that is particularly
well-suited for machine learning and iterative computations.

\balancecolumns
\end{document}
